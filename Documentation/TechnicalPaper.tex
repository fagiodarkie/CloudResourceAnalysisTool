
\newcommand{\filedate}{\today}
\newcommand{\fileversion}{Version 0.1}

%\documentclass{article}
\documentclass{amsart}

%%% The following command loads the amsrefs package, which will be
%%% used to create the bibliography:
%\usepackage[lite]{amsrefs}

%%% The following command defines the standard names for all of the
%%% special symbols in the AMSfonts package, listed in
%%% http://www.ctan.org/tex-archive/info/symbols/math/symbols.pdf
\usepackage{amssymb}

%%% St Mary Road symbols for theoretical computer science
\usepackage{stmaryrd}

%%% mathtools for \mathclap{} used in \sum, and \underbrace{}_n
\usepackage{mathtools}

\usepackage{url}
%\usepackage[pdfborderstyle={/S/U/W 1},hyperfootnotes=false]{hyperref}
%\usepackage[colorlinks=false,pdfborder={0 0 0}]{hyperref}
%\usepackage[colorlinks,pdfborder={0 0 0}]{hyperref}
\usepackage[colorlinks]{hyperref}
%\usepackage{hyperref}

%%% Comment out (or delete) any of these that you don't want to use.
\newcommand{\tensor}{\otimes}
\newcommand{\homotopic}{\simeq}
\newcommand{\homeq}{\cong}
\newcommand{\iso}{\approx}

\DeclareMathOperator{\ho}{Ho}
\DeclareMathOperator*{\colim}{colim}

\newcommand{\N}{\mathbb{N}}
\newcommand{\C}{\mathbb{C}}
\newcommand{\Z}{\mathbb{Z}}

\newcommand{\M}{\mathcal{M}}
\newcommand{\W}{\mathcal{W}}

\newcommand{\bFunc}{\mathbb{B}}
\newcommand{\itilde}{\tilde{\imath}}
\newcommand{\jtilde}{\tilde{\jmath}}
\newcommand{\ihat}{\hat{\imath}}
\newcommand{\jhat}{\hat{\jmath}}

\renewcommand{\emptyset}{\varnothing}

% The following causes equations to be numbered within sections
\numberwithin{equation}{section}

%       Theorem environments
\theoremstyle{plain} %% This is the default, anyway
\newtheorem{thm}[equation]{Theorem}
\newtheorem{cor}[equation]{Corollary}
\newtheorem{lem}[equation]{Lemma}
\newtheorem{prop}[equation]{Proposition}

\theoremstyle{definition}
\newtheorem{defn}[equation]{Definition}

\theoremstyle{remark}
\newtheorem{rem}[equation]{Remark}
\newtheorem{ex}[equation]{Example}
\newtheorem{notation}[equation]{Notation}
\newtheorem{terminology}[equation]{Terminology}

\begin{document}

\title{Behavior types of several JVM instructs}
%\author{}

%%% In the address, show linebreaks with double backslashes:
%\address{Dipartimento di Informatica - Scienza e Ingegneria}

%%% To have the current date inserted, use \date{\today}:
%\date{\filedate, \fileversion}

\maketitle

%%% To include a table of contents, uncomment the following line:
%\tableofcontents

\begin{abstract}
We consider an abstraction of the program $P$ globally defined from 1 to a certain $n$. No checks on the domain of $P$ are made because the bytecode is only valid if all execution path end with a return statement, and in that case the analysis do not consider exceding lines of code.

The JVM state is abstracted with a memory $F$ initially filled with the parameters of the method or program, and a stack frame $S$ initially empty. In case a jump is detected, the equivalence between states are defined as the equivalence between the type of each memory slot or stack address. On the stack $S$ are defined the operations $pop(S): S \mapsto S$, $top(S) \mapsto Int \cup VM \cup \dots$, $push(x, S) \mapsto S$ with the trivial semanthics. The VM states are updated in the environment $\Gamma$.

The behavioural type is given by the function $\bFunc{}_x:F \times S \mapsto \mathbb{B}$ where $\bFunc{}_i(F, S)$ types the program from the instruction $i$ fo the end of the program. This way jumps are easily typed and branches are created in a natural way.

Open problems:
\begin{itemize}
\item when the analysis is performed without actual parameters or with non-specified parameters (eg. expressions which are not in Presburger Arithmetics), backward jumps (may) result in an infinite cycle of branching conditions. A way to manage these conditions should be found. An hypotesis may be to type backward jumps with a slightly different function $\overline{\bFunc{}}$ that is not typed by the system of $\bFunc{}$;\\
\textbf{Solution:} this was no problem from the start. The theorem verifier will take care of recursion, if the recursion is correctly presented (i.e. backwards invocation of $\bFunc{}$ are recursively analysed until a fixed point is reached). Support for least upper bound must be provided for stack, memory and environment.
\item even if computation from created VM is considered not to affect the VM environment, one can still have multiple machines or threads running the same code. While a machine may not know of the VMs created by another one, different threads may access the same global variables, therefore creating race conditions on the access of these variables. In the current version the program is supposed to be non-concurrent, but in the future a way to detect synchronized areas should be considered.\\
\end{itemize}

These rules must be fixed: whenever a backward jump is performed ($\bFunc{}_i = \bFunc{}_j, j < i$), the parameters of $\bFunc{}$ must be replaced with the least upper bound between the parameters at index $i$ and the parameters at index $j$, where in general $i < j \implies lub(P_i, P_j) = P_i$. To accomplish this behavior, we must keep track of the state of VMs, stack and memory in each step $\bFunc{}$. When not specified, $\Gamma_{i+1}$ is supposed to hold the same state as $\Gamma_i$, such as in:\\ $\Gamma_i \vdash \bFunc{}_i(F_i, S_i) = \bFunc{}_{i+1}(F_{i+1}, S_{i+1}) \rhd \Gamma_{i+1} = \Gamma_i$. In the same way, when not specified, the subscript of $\Gamma$ matches the subscript of the typed function $\bFunc{}_i$, such as in:\\
$\Gamma \vdash \bFunc{}_i = \Gamma_i \vdash \bFunc{}_i$.

For the jumps, a function $\cap$ is given to compute the least upper bound for stack, environment and memory.
 
\end{abstract}

\newpage

(T-Program)
\begin{equation*}
\frac{}
{F,\emptyset, \Gamma \vdash P: \bFunc{}_1(F, \emptyset)}
\end{equation*}
\\
(T-Return)
\begin{equation*}
\frac{P[i] = \text{\texttt{return}}}
{\Gamma \vdash \bFunc{}_i(F_i, S_i) = \emptyset}
\end{equation*}
\\
(T-If-Useless)
\begin{equation*}
\frac{\begin{aligned}
P[i] = \text{\texttt{if} $[\text{\textit{cond}}]$ $L$ }, L = i+1,\\
S_{i+1} = \text{pop}(S_i), F_{i+1} = F_i
\end{aligned}}
{\Gamma \vdash \bFunc{}_i(F_i, S_i) = \bFunc{}_{i+1}(F_{i+1}, S_{i+1})}
\end{equation*}
\\
(T-If)
\begin{equation*}\frac{\begin{aligned}
P[i] = \text{\texttt{if} $[\text{\textit{cond}}]$ $L$ }, L \neq 1\\
S_{i+1} = \text{pop}(S_i), F_{i+1} = F_i,\\
S'_{L} = \text{pop}(S_i) \cap S_L, F'_{L} = F_i \cap F_L
\end{aligned}}
{\Gamma_i \vdash \bFunc{}_i(F_i, S_i) = [\text{\textit{cond}}] (\bFunc{}_{L}(F'_{L}, S'_{L})) +
 [\neg \text{\textit{cond}}] (\bFunc{}_{i+1}(F_{i+1}, S_{i+1})) \rhd \Gamma_i \cap \Gamma_L }
\end{equation*}
\\
(T-Goto)
\begin{equation*}
\frac{P[i] = \text{\texttt{goto} $L$}, S'_L = S_i \cap S_L, F'_L = F_i \cap F_L}
{\Gamma_i \vdash \bFunc{}_i(F_i, S_i) = \bFunc{}_{L}(F'_L, S'_L) \rhd \Gamma_i \cap \Gamma_L}
\end{equation*}
\\
(T-New-VM)
\begin{equation*}\frac{\begin{aligned}
P[i] = \text{\texttt{invokevirtual CreateVM} }, \beta \text{ fresh},\\
S_{i+1} = \text{push}(\beta,S_i), F_{i+1} = F_i, \Gamma_{i+1} = \Gamma_i[\beta \mapsto \top]
\end{aligned}}
{\Gamma_i \vdash \bFunc{}_i(F_i, S_i) = \nu \beta \fatsemi \bFunc{}_{i+1}(F_{i+1}, S_{i+1}) \rhd \Gamma_{i+1}}
\end{equation*}
\\
(T-Release-VM)
\begin{equation*}
\frac{\begin{aligned}P[i] = \text{\texttt{invokevirtual ReleaseVM} }, \beta = \text{top}(S_i),\\
\Gamma_i(\beta) \neq \bot, \Gamma_{i+1} = \Gamma_i[\beta \mapsto \bot],
S_{i+1} = \text{pop}(S_i), F_{i+1} = F_i
\end{aligned}}
{\Gamma_i \vdash \bFunc{}_i(F_i, S_i) = \beta^{\checkmark} \fatsemi \bFunc{}_{i+1}(F_{i+1}, S_{i+1}) \rhd \Gamma_{i+1}}
\end{equation*}
\\
(T-Release-VM-Null)
\begin{equation*}
\frac{\begin{aligned}P[i] = \text{\texttt{invokevirtual ReleaseVM} },\\
\beta = \text{top}(S_i), \Gamma_i(\beta) = \bot, S_{i+1} = \text{pop}(S_i), F_{i+1} = F_i
\end{aligned}}
{\Gamma_i \vdash \bFunc{}_i(F_i, S_i) = \bFunc{}_{i+1}(F_{i+1}, S_{i+1})}
\end{equation*}
\\
(T-Load)
\begin{equation*}
\frac{P[i] = \text{\texttt{load }} n, S_{i+1} = \text{push}(F(n), S_i), F_{i+1} = F_i}
{\Gamma \vdash \bFunc{}_i(F_i, S_i) = \bFunc{}_{i+1}(F_{i+1}, S_{i+1})}
\end{equation*}
\\
(T-Store)
\begin{equation*}
\frac{P[i] = \text{\texttt{store} } n, S_{i+1} = \text{pop}(S_i), F_{i+1} = F_i[n \mapsto \text{top}(S_i)]}
{\Gamma \vdash \bFunc{}_i(F_i, S_i) = \bFunc{}_{i+1}(F_{i+1}, S_{i+1})}
\end{equation*}
\\
(T-Integer-Increment)
\begin{equation*}
\frac{P[i] = \text{\texttt{iinc} } idx~n, S_{i+1} = S_i,
F_{i+1} = F_i[idx \mapsto (F_i(idx) + n)]}
{\Gamma \vdash \bFunc{}_i(F_i, S_i) = \bFunc{}_{i+1}(F_{i+1},S_{i+1})}
\end{equation*}

\end{document}
